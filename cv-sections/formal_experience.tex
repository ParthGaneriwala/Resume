%----------------------------------------------------------------------------------------
%	SECTION TITLE
%----------------------------------------------------------------------------------------

\cvsection{Experience}

%----------------------------------------------------------------------------------------
%	SECTION CONTENT
%----------------------------------------------------------------------------------------

\begin{cventries}

%------------------------------------------------

\cventry
{Research Professional}
{ASSIST Research Lab, Florida Institute of Technology}
{Melbourne, FL}
{August 2021 - Present}
{
\begin{cvitems}
\item \textbf{Contributed to competitive research proposals} (technical writing, work packages, milestones, evaluation plans) that resulted in \textbf{\$XX.XM total funded awards}, including NASA ULI Round 8 (up to \$6.9M/3 years) and the Collins/ISU/RTRC/SIFT effort; helped shape technical scope, assurance approach, and partner coordination.
\item \textbf{Delivered assurance/verification research} across formal methods, run-time assurance, ML, robotics, and cyber-physical systems; owned key technical workstreams from problem formulation through implementation, evaluation, and publication-quality reporting.
\item \textbf{NASA ULI (Round 8): Trustworthy increasingly autonomous aviation safety systems} — developed/implemented \textbf{[your specific contribution: e.g., formal models, assurance cases, verification experiments, datasets, evaluation metrics]}, coordinating technical interfaces with multi-university/industry partners (Penn State, UF, Collins Aerospace, ResilienX, etc.).
\item \textbf{DARPA-funded AR/VR cyber-sickness modeling (Collins/ISU/RTRC/SIFT)} — built \textbf{[formal modeling / validation / analysis]} of human cognitive-behavior representations; produced \textbf{[artifacts: models, code, experiments, reports]} used in partner reviews.
\item \textbf{LLM translation for cognitive architectures (with Penn State)} — designed and evaluated \textbf{[pipeline/benchmarks]} for LLM-assisted translation to improve communication/knowledge transfer between cognitive systems; \textbf{[add measurable outcome if available]}.
\item \textbf{Autonomous aviation perception + data} — developed a vision pipeline to \textbf{detect/label taxiway markings} and contributed to \textbf{AssistTaxi} dataset creation/analysis for runway/taxiway understanding; \textbf{[add metrics: mAP/IoU, labeling throughput, dataset size]}.
\item \textbf{Cognitive-enhanced agent autonomy} — implemented and tested \textbf{[planning/control/assurance]} for automated piloting scenarios emphasizing safe takeoff/landing and dense-traffic operations; \textbf{[simulation results or demo milestones]}.
\item \textbf{Advised and mentored} researchers and students: advised a Ph.D. student on transfer learning + assurance for vision-based autonomous aircraft (NASA-funded); mentored undergraduate/high-school teams leading to \textbf{conference publications}.
\item \textbf{Teaching support} — designed quizzes/homework projects for Python, DB Systems, Web Apps, Big Data, and Software Metrics; supported instruction and student outcomes.
\item \textbf{Recognition/leadership} — Outstanding Student of the Year (Honors Convocation 2025), Phi Kappa Phi, President (Florida Tech Badminton Club).
\end{cvitems}
}
\cventry
{Software Engineering Intern}
{Software and Systems Engineering, \href{https://www.avidyne.com/}{Avidyne}}
{\textbf{Melbourne, FL}}
{\textbf{May 2025 – Present}}
{
\begin{cvitems}
  \item Designed and executed \textbf{end-to-end (E2E) test and evaluation (E2TE)} workflows for \textbf{aviation simulation software}, increasing test coverage across navigation, communication, and flight display systems by 25\%.
  \item Developed, integrated, and debugged \textbf{C-based flight software modules} in the avionics stack, ensuring compliance with real-time, safety-critical, and DO-178C guidelines.
  \item Created and optimized \textbf{50+ system-level and flight-specific test cases} in simulation environments, reducing verification cycle time by 15\%.
  \item Automated regression and validation processes by writing \textbf{Visual Basic and C test scripts}, accelerating simulation turnaround by 20\%.
  \item Utilized \textbf{industry-standard tools} — Perforce (P4), Visual Studio, and proprietary avionics simulation frameworks — to streamline development and validation pipelines.
  \item Executed \textbf{hardware-in-the-loop simulations}, diagnosing and resolving execution issues to improve simulation-to-aircraft fidelity by 10\%.
  \item Contributed to flight code development for Avidyne’s \href{https://quantum.avidyne.com/}{\textbf{Quantum Open Avionics Platform}}, supporting rapid prototyping of \textbf{customizable, next-generation avionics solutions}.
\end{cvitems}
}
\cventry
{Graduate Research Assistant}
{\href{https://research.fit.edu/l3hiai/}{L3Harris Institute for Assured Information}, \href{www.fit.edu}{Florida Institute of Technology}}
{\textbf{Melbourne, FL}}
{\textbf{May 2024 – July 2024}}
{
\begin{cvitems}
\item Developed a \textbf{decentralized framework} enabling \textbf{multiple autonomous agents}—robotic dogs, drones, and mobile robots—to coordinate, communicate, and reach shared goals.
\item Collaborated with \textbf{developers and professors} to rigorously test the system in both \textbf{simulation} and \textbf{real-world environments}.
\end{cvitems}
}

\cventry
{Graduate Research Assistant}
{\href{https://www.fit.edu/faculty-profiles/s/slhoub-khaled-/}{IRI Research},  \href{www.fit.edu}{Florida Institute of Technology}}
{\textbf{Melbourne, FL}}
{\textbf{May 2023 - August 2023}}
{
\begin{cvitems}
% \item ASSIST Research Lab focuses on the design and development of assurance frameworks for mission-critical, safety-critical, and security-critical systems.
\item Proposed and \textbf{implemented a framework using AI language models} to automatically extract software requirements from source code.
% \item Proposed a novel automated framework that extracts software requirements directly from source code using LLMs APIs.
\item Supervised and coordinated with undergraduate students towards the development process.
\end{cvitems}
}
\end{cventries}