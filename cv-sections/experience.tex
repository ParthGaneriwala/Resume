%----------------------------------------------------------------------------------------
%	SECTION TITLE
%----------------------------------------------------------------------------------------

\cvsection{Experience}

%----------------------------------------------------------------------------------------
%	SECTION CONTENT
%----------------------------------------------------------------------------------------

\begin{cventries}

%------------------------------------------------
\cventry
{Research Professional}
{\href{https://research.fit.edu/assist-lab/}{ASSIST Research Lab}, \href{www.fit.edu}{Florida Institute of Technology}}
{\textbf{Melbourne, FL}}
{\textbf{August 2021 - Present}}
{
\begin{cvitems}
% \item ASSIST Research Lab focuses on the design and development of assurance frameworks for mission-critical, safety-critical, and security-critical systems.
\item Working with a team of research professionals for formal methods of verification and run-time assurance, ML, IoT, robotics, and cyber security.
\item Collaborating on NASA’s \textit{\textbf{University Leadership Initiative (Round 8)}} as part of a \textbf{Florida Tech}-led multi-university/industry team to develop a framework for \textbf{trustworthy, increasingly autonomous aviation safety systems}; partners include \textit{\textbf{Penn State, NC A\&T State, UF, Stanford, Santa Fe College, Uni of New Mexico, Collins Aerospace, and ResilienX}}; part of awards totaling up to \$20.7M over three years.
% \item Collaborated with faculty and published research involving IoT, Cyber Security, Formal Verification, and  Cognitive Agents for promoting high-quality software development and evaluating agent-oriented methodologies.
\item Collaborating with \textit{\textbf{Collins Aerospace, Iowa State University, Raytheon Technologies Research Center (RTRC), and Smart Information Flow Technologies (SIFT)}}, funded by \textit{\textbf{DARPA}} with the task of formally modeling human cognitive behavior representation with respect to cyber-sickness in AR/VR systems. 
\item Advising a Ph.D. student on transfer learning, automated data labeling, and assurance frameworks for vision-based classification in autonomous aircraft systems funded by \textit{\textbf{NASA}}, addressing safety and reliability challenges in aviation technologies.
\item Collaborating with \textit{\textbf{Penn State University}} on the application of \textbf{Large Language Model (LLM) translation} for cognitive architectures, focusing on enhancing the integration of LLMs to facilitate communication and knowledge transfer within cognitive systems.
\item Contributed with \textit{\textbf{Critical Frequency Design}}, funded by \textit{\textbf{Naval Air Systems Command}}, with the task of developing a modeling approach for designing, maintaining, and supporting air and sea platform fiber optic communications technology.
\item Collaborated with \textit{\textbf{Rockwell Collins and Soar Tech}}, funded by \textit{\textbf{NASA}} with the task of formally verifying the autonomous agent to assure safety as well as the logical correctness of the safety-critical system.
\item Collaborated with \textbf{professors on the development of research proposals} on diverse topics, including cognitive agents on human behavior, the assurance of artificial intelligence in safety-critical systems, and the fine-tuning of LLMs for domain-specific queries.
\item Investigated the development of a \textbf{cognitive-enhanced agent} for automatically piloting aircraft in dense urban environments which emphasized safe and reliable takeoff/landing among aerial traffic without human intervention.
% \item Proposed a novel framework extracting software requirements directly from source code using AI language models.
\item Developed an \textbf{autonomous aircraft perception system} for accurately detecting and labeling line markings on an airport taxiway.
\item Presented \textbf{AssistTaxi}, a novel dataset for runway and taxiway analysis, contributing to autonomous operations.
% \item Addressed the lack of formal representations for function-based modeling and reasoning in SysML, introducing XML-based reasoning and XMLSlim, an algorithm optimizing XML files, demonstrating a 60\% reduction in file size while preserving meaning.
\item Advised \textbf{5 groups of computer science students} on the design, development, and deployment of software for their senior projects
\item Mentored \textbf{undergraduate} and \textbf{high school students} on machine learning engineering approaches in the aerospace and systems domains, leading to \textbf{conference publications} that addressed real-world challenges in these fields.
\item Assisted with the \textbf{formulation of quizzes and homework projects} for the courses: Python, Database Systems, Web Applications, Big Data and Management, and Software Metrics.
\item Recognized with multiple honors and leadership roles, including \textbf{Outstanding Student of the Year} at Honors Convocation 2025, \textbf{Inducted Member of Phi Kappa Phi}, and \textbf{President of the Florida Tech Badminton Club}.
\end{cvitems}
}
\cventry
{Software Engineering Intern}
{Software and Systems Engineering, \href{https://www.avidyne.com/}{Avidyne}}
{\textbf{Melbourne, FL}}
{\textbf{May 2025 – Present}}
{
\begin{cvitems}
  \item Designed and executed \textbf{end-to-end (E2E) test and evaluation (E2TE)} workflows for \textbf{aviation simulation software}, increasing test coverage across navigation, communication, and flight display systems by 25\%.
  \item Developed, integrated, and debugged \textbf{C-based flight software modules} in the avionics stack, ensuring compliance with real-time, safety-critical, and DO-178C guidelines.
  \item Created and optimized \textbf{50+ system-level and flight-specific test cases} in simulation environments, reducing verification cycle time by 15\%.
  \item Automated regression and validation processes by writing \textbf{Visual Basic and C test scripts}, accelerating simulation turnaround by 20\%.
  \item Utilized \textbf{industry-standard tools} — Perforce (P4), Visual Studio, and proprietary avionics simulation frameworks — to streamline development and validation pipelines.
  \item Executed \textbf{hardware-in-the-loop simulations}, diagnosing and resolving execution issues to improve simulation-to-aircraft fidelity by 10\%.
  \item Contributed to flight code development for Avidyne’s \href{https://quantum.avidyne.com/}{\textbf{Quantum Open Avionics Platform}}, supporting rapid prototyping of \textbf{customizable, next-generation avionics solutions}.
\end{cvitems}
}
\cventry
{Graduate Research Assistant}
{\href{https://research.fit.edu/l3hiai/}{L3Harris Institute for Assured Information}, \href{www.fit.edu}{Florida Institute of Technology}}
{\textbf{Melbourne, FL}}
{\textbf{May 2024 – July 2024}}
{
\begin{cvitems}
\item Developed a \textbf{decentralized framework} enabling \textbf{multiple autonomous agents}—robotic dogs, drones, and mobile robots—to coordinate, communicate, and reach shared goals.
\item Collaborated with \textbf{developers and professors} to rigorously test the system in both \textbf{simulation} and \textbf{real-world environments}.
\end{cvitems}
}

\cventry
{Graduate Research Assistant}
{\href{https://www.fit.edu/faculty-profiles/s/slhoub-khaled-/}{IRI Research},  \href{www.fit.edu}{Florida Institute of Technology}}
{\textbf{Melbourne, FL}}
{\textbf{May 2023 - August 2023}}
{
\begin{cvitems}
% \item ASSIST Research Lab focuses on the design and development of assurance frameworks for mission-critical, safety-critical, and security-critical systems.
\item Proposed and \textbf{implemented a framework using AI language models} to automatically extract software requirements from source code.
% \item Proposed a novel automated framework that extracts software requirements directly from source code using LLMs APIs.
\item Supervised and coordinated with undergraduate students towards the development process.
\end{cvitems}
}
%------------------------------------------------
% \cventry
% {Summer Intern}
% {RoughPaper Technologies}
% {Dubai, U.A.E.}
% {June 2021 - August 2021}
% {
% \begin{cvitems}
% % \item Interned at a software-based company skilled in business optimization, media creation, website, and software development.
% \item Developed products of a Food Delivery Web Application using React and Node.js. 
% % \item Created user interface in HTML/JQuery and led project integration to create a functional search engine.
% \item Explored the use of Gather. town which is centered around fully customizable spaces.
% \item Aided the software team with designing client-specific web-based solutions.
% \end{cvitems}
% }

%------------------------------------------------

% \cventry
% {Team Lead}
% {Team IFOR \& Team IORTA, BITS Pilani}
% {Dubai, U.A.E.}
% {January 2019 - July 2021}
% {
% \begin{cvitems}
% \item Guided students from different departments to design and build autonomous aerial robotic systems.
% % \item Project Lead for Autonomous and Manual Testing of HexCopter and QuadCopter. 
% \item Formulated the autonomous code for the drones and built a prototype which was tested in the field. 
% % \item Working with a team and designing autonomous drones from social distancing applications.
% % \item Designed Neural Networks to classify if a certain subject in the field of view of a mobile robot is following social distancing norms.
% % \item Designed an autonomous drone and integrated it with washing mechanisms and used Window Detection with OpenCV. 
% % \item Finalists for the BITS Innovation Competition 2019.
% \end{cvitems}
% }

%------------------------------------------------

% \cventry
% {Graphic Design Intern}
% {Admissions Office of BITS Pilani}
% {Dubai, U.A.E.}
% {January 2020 - May 2020}
% {
% \begin{cvitems}
% \item Worked as a Professional Assistant for the Admission Department of BITS Pilani, Dubai campus, assisting them with the admission process for the year 2020-21.
% % \item Assisted with the design of print materials such as flyers and ads for digital sales and marketing campaigns.
% \end{cvitems}
% }

%------------------------------------------------

% \cventry
% {Senior Reporter}
% {Editorial Board, BITS Pilani}
% {Dubai, U.A.E}
% {August 2019 - June 2021}
% {
% \begin{cvitems}
% % \item The Editorial Board is responsible for Perspectives, a college newsletter, and @bitsdubai, the official college magazine.
% \item Drafted articles and reported daily events in the university to the newsletter.
% \item Undertook the recruiting process and reviewed submitted entries for @bitsdubai, the official college magazine.
% \end{cvitems}
% }
% \cventry
% {Project Lead}
% {Team IORTA, BITS Pilani}
% {Dubai, U.A.E.}
% {January 2019 - December 2019}
% {
% \begin{cvitems}
% % \item Window Cleaning Drone for the BITS Innovation Competition 2019. Finalists for the BITS Innovation Competition 2019.
% \item Designed an autonomous drone and integrated it with washing mechanisms and used Window Detection with OpenCV. 
% \item Finalists for the BITS Innovation Competition 2019. 
% \end{cvitems}
% }

%------------------------------------------------

% \cventry
% {Summer Intern}
% {Ganeriwala \& Ganeriwal Chartered Accountants}
% {Kolkata, West Bengal}
% {June 2018 - August 2018}
% {
% \begin{cvitems}
% % \item Database Building/Management for the company of Chartered Accountants.
% \item Arranged the office server system and built a database to assess the client dues and all the services listed under one umbrella.
% % \item Arranged the office server system and created a database system for all clients and services provided.
% \end{cvitems}
% }
%  \cventry
%     {Computer Science Lecturer}
%     {Morgan State University}
%     {Baltimore, MD}
%      {July 2012 - August 2014}
%     {
%     \begin{cvitems}
%     \item Taught Computer Literacy, Technology, Human and Social Values course to over 800 undergraduate students, resulting in a commendation from the department chairperson.
%   \item Assisted in redesign of curriculum, and team-taught Scientific Visualization course to graduate students
%     \item Assisted students in thesis writing centered on parallel computing using C++, CUDA, and MPI 
%     \item Taught students to use productivity and desktop publishing software to prepare technical reports, documents, spreadsheets, websites, and databases
%     \item Taught programming in C++, Java, and JavaScript, web development, and robotics using the Finch robot to underrepresented students\\
% \end{cvitems}
%     }

%------------------------------------------------
%  \cventry
% {Computer Programmer}
% {University of Illinois Data Sciences Summer Institute}
% {Urbana-Champaign, IL}
%  {May 2011 - June 2011}
% {     
% \begin{cvitems}
% \item Interned as a programmer at the Multimodal Information Access and Synthesis (MIAS) Center at the Data Science Summer Institute and created an expert search engine in collaboration with the Department of Homeland Security (DHS)
% \item Created web crawler using Python programming language
% \item Preprocessed large volumes of data(13,609 documents) for 4,445 experts
% \item Created user interface in HTML/JQuery and led project integration to create functional search engine\\
% \end{cvitems}
%     }

%------------------------------------------------

\end{cventries}